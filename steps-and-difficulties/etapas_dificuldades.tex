\documentclass[a4paper, 12pt]{article}

\usepackage[brazil]{babel}
\usepackage[utf8]{inputenc}
\usepackage{hyperref} % criar hyperlinks
\usepackage{listings} % utilizado para dar highlight em código
\usepackage{xcolor}

\lstdefinestyle{customc}{
  belowcaptionskip=1\baselineskip,
  breaklines=true,
  frame=single,
  xleftmargin=\parindent,
  language=C,
  showstringspaces=false,
  basicstyle=\footnotesize\ttfamily,
  keywordstyle=\bfseries\color{green},
  commentstyle=\itshape\color{purple},
  identifierstyle=\color{blue},
  stringstyle=\color{orange},
}

\title{Etapas e dificuldades}
\author{Victor Vieira}
\author{Arthur Cicuto}
\date{\today}

\begin{document}

\maketitle

\section{Módulo do Ônibus}

\subsection{Hardware}

HM-10 - Bluetooth 4.0 BLE module

\subsection{Configuração}

Comandos AT.

Descrever aqui as etapas.

Código para configurar um arduino

\lstinputlisting[style=customc]{codes/arduino-code.ino}

\subsection{Referências}

\href{ftp://imall.iteadstudio.com/Modules/IM130614001_Serial_Port_BLE_Module_Master_Slave_HM-10/DS_IM130614001_Serial_Port_BLE_Module_Master_Slave_HM-10.pdf}{Datasheet}

\section{Módulo do ponto de ônibus}

teste

\section{Aplicativo}

teste

\end{document}